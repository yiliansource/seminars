\documentclass[letterpaper, 11pt]{article}

\usepackage[english]{babel}
\usepackage{lipsum}
\usepackage[dvipsnames]{xcolor}
\usepackage{
  amsmath, amsthm, amssymb, mathtools, dsfont, units, stmaryrd,
  graphicx, wrapfig, subfig, float,                         
  listings, color, inconsolata, pythonhighlight,            
  fancyhdr, enumerate, enumitem, framed
}
\usepackage[unicode,colorlinks=false,pagebackref=false]{hyperref} 
\usepackage[left=1.35in, right=1.35in, top=1.0in, bottom=.9in, headsep=.2in, footskip=0.35in]{geometry}
\usepackage[bottom]{footmisc}
\usepackage{anyfontsize}
\usepackage{mdframed, color}
\usepackage{cleveref}

\usepackage{tikz}
\usetikzlibrary{cd}

\renewcommand{\baselinestretch}{1.2}
\setlength{\parskip}{1.3mm}
\setlength{\parindent}{0mm}
\allowdisplaybreaks

\setlist[enumerate]{itemsep=0mm,topsep=1mm}


\usepackage[font={it,footnotesize}]{caption}
% \usepackage{titlesec}
% \titleformat{\section}{\large\bfseries}{\thesection. }{0em}{}
% \titleformat{\subsection}{\normalsize\bfseries\selectfont}{\thesubsection\;\;\;}{0em}{}
\setlist[itemize]{wide=0pt, leftmargin=16pt, labelwidth=10pt, align=left}



% \usepackage[subfigure]{tocloft}
% \setlength{\cftbeforesecskip}{.9ex}
% \cftsetindents{section}{0em}{0em}
% \renewcommand{\cfttoctitlefont}{\large\bfseries}

% \makeatletter
% \renewcommand{\cftsecpresnum}{\begin{lrbox}{\@tempboxa}}
% \renewcommand{\cftsecaftersnum}{\end{lrbox}}
% \makeatother

% \graphicspath{{Images/}{../Images/}}



% \newcommand\numberthis{\addtocounter{equation}{1}\tag{\theequation}}
% \let\b\mathbf
% \let\bg\boldsymbol
% \let\mc\mathcal

\usepackage[mathscr]{euscript}

\newcommand{\M}{\mathcal{M}}
\newcommand{\R}{\mathbb{R}}
\newcommand{\C}{\mathbb{C}}
\newcommand{\D}{\mathbb{D}}
\newcommand{\K}{\mathbb{K}}
\newcommand{\Q}{\mathbb{Q}}
\newcommand{\Z}{\mathbb{Z}}
\newcommand{\T}{\mathbb{T}}
\newcommand{\N}{\mathbb{N}}
\newcommand{\F}{\mathcal{F}}
\newcommand{\El}{\mathcal{L}}
\newcommand{\1}{\mathds{1}}
\newcommand{\dd}{\mathop{}\!\mathrm{d}}
\newcommand{\clspan}{\mathrm{clspan}}
\newcommand{\mapping}[3]{
  \ensuremath{
    #1 : \left\{
    \begin{array}{@{\;}r@{\hspace{0.4cm}}c@{\hspace{0.4cm}}l@{}}
      #2 \\
      #3
    \end{array}
    \right.
  }
}
\newcommand{\cl}[1]{\overline{#1}}
\newcommand{\restr}[2]{{% we make the whole thing an ordinary symbol
  \left.\kern-\nulldelimiterspace % automatically resize the bar with \right
  #1 % the function
  \littletaller % pretend it's a little taller at normal size
  \right|_{#2} % this is the delimiter
  }}

\newcommand{\littletaller}{\mathchoice{\vphantom{\big|}}{}{}{}}


\DeclareMathOperator*{\argmax}{argmax}
\DeclareMathOperator*{\argmin}{argmin}
\DeclareMathOperator*{\spann}{Span}		
\DeclareMathOperator*{\bias}{Bias}		
\DeclareMathOperator*{\ran}{ran}			
\DeclareMathOperator*{\dv}{d\!}
\DeclareMathOperator*{\diag}{diag}		
\DeclareMathOperator*{\trace}{trace}	
\DeclareMathOperator*{\supp}{supp}
\DeclareMathOperator*{\Hol}{Hol}

\theoremstyle{definition}
\newtheorem{theorem}{Theorem}
\newtheorem{proposition}[theorem]{Proposition}
\newtheorem{lemma}[theorem]{Lemma}
\newtheorem{corollary}[theorem]{Corollary}
\newtheorem{definition}[theorem]{Definition}
\newtheorem{example}[theorem]{Example}
\newtheorem{remark}[theorem]{Remark}

\newenvironment{innerproof}
 {\renewcommand{\qedsymbol}{}\proof}
 {\endproof}

% \AtBeginEnvironment{exercise}{%
%   \pushQED{\qed}\renewcommand{\qedsymbol}{$\sslash$}
% }
% \AtEndEnvironment{exercise}{\popQED\endexercise}

% \AtBeginEnvironment{solution}{%
%   \pushQED{\qed}\renewcommand{\qedsymbol}{$\blacksquare$}
% }
% \AtEndEnvironment{solution}{\popQED\endsolution}


% \pagestyle{fancy}
% \renewcommand{\footrulewidth}{0pt}
% \renewcommand{\headrulewidth}{0pt}

% \setlength{\headheight}{14pt}
% \renewcommand{\sectionmark}[1]{\markright{#1} }
% \fancyhead[R]{\small\textit{\nouppercase{\rightmark}}}
% \fancyhead[L]{Left Header}

\usepackage{bold-extra}

\begin{document}

% \title{\textbf{The Smith factorization}}
\author{\normalsize Ian Hornik}
\date{\vspace{-0.8em}\normalsize\today}

% \maketitle

% \begin{center}
%   {\LARGE\scshape\bfseries{The Corona Theorem}} \\
%   \scshape{Ian Hornik}
% \end{center}

$(s_n)_{n=0}^\infty$ Folge in $\R$ (Daten). Dann $\exists \mu$ pos. Maß (und alle Momente existieren), $\forall n \in \N : s_n = \int_\R t^n \dd \mu$ (Lösung).

Idee: Man baut sich ein Operator-Modell und stellt fest, dass ein gewisser Teil von diesem Operatormodell schon durch die Daten festgelegt ist. Wir wollen also diesen Teil erweitern und damit eine Lösung erhalten.

\begin{enumerate}
  \item Sei $\mu$ ein positives Maß af $\R$ mit $\forall n \in \N : \int_R \vert t \vert^n \dd \mu < \infty$.
  $$ \rightsquigarrow \langle L^2(\mu), M_t, 1 \rangle, $$
  wobei $M_t$ den Multiplikationsoperator mit $t$ bezeichnet und $1$ die konstante $1$-Funktion. Wir haben also einen Hilbertraum $(L^2(\mu))$, einen selbstadjungierten Operator $(M_t)$ und das Element $1$. Dieses ist ``nicht nur'' ein Element, sondern ein erzeugendes Element in dem Sinn, dass $\mathrm{cls} \{ (M_t - z)^{-1}1 : z \in \C \setminus \R \} = L^2(\mu)$.

  Aus diesem Modell können wir uns $\mu$ rekonstruieren als $E_{1,1}$, wobei $E$ das Spektralmaß von $M_t$ ist.

  Wir sehen also: Hat man ein Tupel $\langle H, A, v \rangle$ bestehend aus einem Hilbertraum $H$, einem selbstadjungierten Operator $A$ und einem Element $v$, so kann man ``das $\mu$ rekonstruieren'', nämlich genau als Maß $E_{v,v}$ (endl. pos. Maß auf $\R$), wobei $E$ das Spektralmaß von $A$ bezeichnet.

  \item Sei nun angenommen wir haben Daten und eine Lösung, also $\mu$ ein positives Maß auf $\R$ und $\forall n \in \N : \int_\R \vert t \vert^n \dd \mu < \infty$. Sei $s_n \coloneqq \int_\R t^n \dd \mu$. Dann erhalten wir auf $\C[z] \subseteq L^2(\mu)$ ein Skalarprodukt
  $$ (p, q) \coloneqq \int_\R p \bar{q} \dd \mu. $$
  Dann ist
  $$ (\sum_n \alpha_n t^n, \sum_m \beta_m t^m) = \sum_{n,m} s_{n+m} \alpha_n \bar{\beta_m}. $$
  Man kennt also $\mathrm{cl} \C[t] / (\mu-\textrm{fü}) \subseteq L^2(\mu)$ sogar \emph{nur} aus $(s_n)_{n=0}^\infty$. Wir kennen also
  $$ M_t \vert_{\C[t]} : \C[t] \to \C[t] $$
  (was wir mit $S_0$ bezeichnen) bereits \emph{nur} aus ?? (Daten).

  \item Sei $(s_n)_{n=0}^\infty$ Folge reeller Zahlen.
  $$ \rightsquigarrow \langle \C[t], S_0, 1 \rangle. $$
  Also einen linearen Raum $(\C[t])$, eine lineare Abbildung $(S_0)$ und ein Element $(1)$. Wir definieren also eine symmetrische Bilinearform
  $$ (\sum_n \alpha_n t^n, \sum_m \beta_m t^m) \coloneqq \sum_{n,m} s_{n+m} \alpha_n \bar{\beta_m}. $$
  Wir hätten allerdings gerne ein pos. semidef. Skalarprodukt, bzgl. dem $S_0$ symmetrisch ist. Das geht also z.B. falls $\forall N : \det (s_{n+m})_{n,m=0}^N \geq 0$.

  \item Für die Folge
  $$ s_n \coloneqq \int_\R t^n \dd \mu $$
  gilt $\forall N : \det (s_{n+m})_{n,m=0}^N \geq 0$.

  \item Sei $(s_n)_{n=0}^\infty$ eine Folge reeler Zahlen mit $\forall N : \det (s_{n+m})_{n,m=0}^N \geq 0$. Betrachte
  $$ \langle \C[t], S_0, 1 \rangle $$
  und eine Erweiterung
  $$ \langle H, A, u \rangle, $$
  in dem Sinne, dass $\iota : \C[t] \to H$ isometrisch und $\iota \circ S_0 = A \circ \iota$ (beachte $\iota(\mathrm{dom} S_0) \subseteq \mathrm{dom} A$) und $u = \iota(1)$. Setze $\mu \coloneqq E_{u, u}$, wobei $E$ das Spektralmaß von $A$ ist. Dann ist $\mu$ ein positives Maß und es gilt für $n \in \N$
  \begin{align*}
    s_n &= (t^n, 1)_{\C[t]} = (S_0^n 1, 1)_{\C[t]} = ([\iota \circ S_0^n] 1, \iota(1))_H = (A^n \iota(1), \iota(1)) = \\
    &= (A^n u, u) = \int_\R t^n \dd E_{u, u}.
  \end{align*}
\end{enumerate}

Wir formulieren also die Fragestellung neu ...
\begin{itemize}
  \item Für welche $(s_n)_{n=0}^\infty$ gibt es eine Lösung?
  \item Wenn es eine Lösung, beschreibe alle Lösungen.
\end{itemize}
... und beantworten sie.

\emph{Existenz von Lösungen}: Gegeben $(s_n)_{n=0}^\infty$ eine Folge reeller Zahlen. Dann gibt es eine positives Maß $\mu$ auf $\R$ mit $s_n = \int_\R t^n \dd \mu$ genau dann wenn für alle $N \in \N$ gilt $\det (s_{n+m})_{n,m=1}^N \geq 0$. (Dies haben wir in (5) gezeigt.)

\emph{Beschreibe alle Lösungen}: Es gibt eine Formel die Bijektion zwischen einer Menge von Parametern und Lösungsmenge herstellt (falls die Lösungsmenge mindestens zwei Elemente hat).
$$ \int_\R = \frac{1}{z - t} \dd \mu(t) = \frac{A(z) \tau(z) + B(z)}{C(z) \tau(z) + D(z)}, $$
wobei $A, B, C, D$ auf $\C$ analytische Funktionen sind welche durch $(s_n)_{n=0}^\infty$ gegeben sind. Der Parameter $\tau$ Parameter durchläuft $\{ f : \C^+ \to \C : \textrm{analytisch und } \Im f \geq 0 \}$.

\end{document}