\documentclass[letterpaper, 11pt]{article}

\usepackage[english]{babel}
\usepackage{lipsum}
\usepackage[dvipsnames]{xcolor}
\usepackage{
  amsmath, amsthm, amssymb, mathtools, dsfont, units, stmaryrd,
  graphicx, wrapfig, subfig, float,                         
  listings, color, inconsolata, pythonhighlight,            
  fancyhdr, enumerate, enumitem, framed
}
\usepackage[unicode,colorlinks=false,pagebackref=false]{hyperref} 
\usepackage[left=1.35in, right=1.35in, top=1.0in, bottom=.9in, headsep=.2in, footskip=0.35in]{geometry}
% \usepackage[bottom]{footmisc}
\usepackage{anyfontsize}
\usepackage{mdframed, color}
\usepackage{cleveref}

\usepackage{tikz}
\usetikzlibrary{cd}

\renewcommand{\baselinestretch}{1.2}
\setlength{\parskip}{1.3mm}
\setlength{\parindent}{0mm}
\setlength{\skip\footins}{0.7cm}
\allowdisplaybreaks

\setlist[enumerate]{itemsep=0mm,topsep=1mm}


\usepackage[font={it,footnotesize}]{caption}
% \usepackage{titlesec}
% \titleformat{\section}{\large\bfseries}{\thesection. }{0em}{}
% \titleformat{\subsection}{\normalsize\bfseries\selectfont}{\thesubsection\;\;\;}{0em}{}
\setlist[itemize]{wide=0pt, leftmargin=16pt, labelwidth=10pt, align=left}



% \usepackage[subfigure]{tocloft}
% \setlength{\cftbeforesecskip}{.9ex}
% \cftsetindents{section}{0em}{0em}
% \renewcommand{\cfttoctitlefont}{\large\bfseries}

% \makeatletter
% \renewcommand{\cftsecpresnum}{\begin{lrbox}{\@tempboxa}}
% \renewcommand{\cftsecaftersnum}{\end{lrbox}}
% \makeatother

% \graphicspath{{Images/}{../Images/}}



% \newcommand\numberthis{\addtocounter{equation}{1}\tag{\theequation}}
% \let\b\mathbf
% \let\bg\boldsymbol
% \let\mc\mathcal

\usepackage[mathscr]{euscript}

\newcommand{\M}{\mathcal{M}}
\newcommand{\R}{\mathbb{R}}
\newcommand{\C}{\mathbb{C}}
\newcommand{\D}{\mathbb{D}}
\newcommand{\K}{\mathbb{K}}
\newcommand{\Q}{\mathbb{Q}}
\newcommand{\Z}{\mathbb{Z}}
\newcommand{\T}{\mathbb{T}}
\newcommand{\N}{\mathbb{N}}
\newcommand{\F}{\mathcal{F}}
\newcommand{\El}{\mathcal{L}}
\newcommand{\1}{\mathds{1}}
\newcommand{\diff}{\mathop{}\!\mathrm{d}}
\newcommand{\clspan}{\mathrm{clspan}}
\newcommand{\mapping}[3]{
  \ensuremath{
    #1 : \left\{
    \begin{array}{@{\;}r@{\hspace{0.4cm}}c@{\hspace{0.4cm}}l@{}}
      #2 \\
      #3
    \end{array}
    \right.
  }
}
\newcommand{\cl}[1]{\overline{#1}}
\newcommand{\restr}[2]{{% we make the whole thing an ordinary symbol
  \left.\kern-\nulldelimiterspace % automatically resize the bar with \right
  #1 % the function
  \littletaller % pretend it's a little taller at normal size
  \right|_{#2} % this is the delimiter
  }}

\newcommand{\littletaller}{\mathchoice{\vphantom{\big|}}{}{}{}}

\newcommand{\Wirtz}[1][]{\frac{\partial #1}{\partial z}}
\newcommand{\wirtz}{\partial}

\newcommand{\Wirtzbar}[1][]{\frac{\partial #1}{\partial \bar{z}}}
\newcommand{\wirtzbar}{\overline{\partial}}

\DeclareMathOperator*{\argmax}{argmax}
\DeclareMathOperator*{\argmin}{argmin}
\DeclareMathOperator*{\spann}{Span}		
\DeclareMathOperator*{\bias}{Bias}		
\DeclareMathOperator*{\ran}{ran}			
\DeclareMathOperator*{\dv}{d\!}
\DeclareMathOperator*{\diag}{diag}		
\DeclareMathOperator*{\trace}{trace}	
\DeclareMathOperator*{\supp}{supp}
\DeclareMathOperator*{\Hol}{Hol}

\theoremstyle{definition}
\newtheorem{theorem}{Theorem}
\newtheorem{proposition}[theorem]{Proposition}
\newtheorem{lemma}[theorem]{Lemma}
\newtheorem{corollary}[theorem]{Corollary}
\newtheorem{definition}[theorem]{Definition}
\newtheorem{example}[theorem]{Example}
\newtheorem{remark}[theorem]{Remark}

\newenvironment{innerproof}
 {\renewcommand{\qedsymbol}{}\proof}
 {\endproof}

% \AtBeginEnvironment{exercise}{%
%   \pushQED{\qed}\renewcommand{\qedsymbol}{$\sslash$}
% }
% \AtEndEnvironment{exercise}{\popQED\endexercise}

% \AtBeginEnvironment{solution}{%
%   \pushQED{\qed}\renewcommand{\qedsymbol}{$\blacksquare$}
% }
% \AtEndEnvironment{solution}{\popQED\endsolution}


% \pagestyle{fancy}
% \renewcommand{\footrulewidth}{0pt}
% \renewcommand{\headrulewidth}{0pt}

% \setlength{\headheight}{14pt}
% \renewcommand{\sectionmark}[1]{\markright{#1} }
% \fancyhead[R]{\small\textit{\nouppercase{\rightmark}}}
% \fancyhead[L]{Left Header}

\usepackage{bold-extra}

\begin{document}

% \title{\textbf{The Smith factorization}}
\author{\normalsize Ian Hornik}
\date{\vspace{-0.8em}\normalsize\today}

% \maketitle

\begin{center}
  {\LARGE\scshape\bfseries{The Corona Theorem}} \\
  \scshape{Ian Hornik}
\end{center}

These are my notes for my talk in the winter semester of 2024, in the analysis seminar at the Technical University of Vienna.

% Content Summary:
% §0: a brief summary of what we want to, concluding with the statement of the equivalent theorem we want to prove.
% §1: First steps - reduce to holomorphic functions on cl(D) and then solve via smooth functions.
% §2: Wirtinger operators - overview and some properties, along with Koszul complex and its application to the proof.
% §3: Hardy spaces - overview and some properties, along with the dualisation lemma and application the proof.
% §4: Integral estimates - show the required estimates and apply them to the proof.

We will denote the space of all complex-valued, bounded, analytic functions on the unit disk $\D$ as $H^\infty$. Equipped with the supremum norm $\Vert \cdot \Vert_\infty$ this space becomes a commutative Banach algebra. The space of all multiplicative, bounded, linear functionals on $H^\infty$ not identically zero is denoted $\Delta(H^\infty)$ and is called the \emph{Gelfand space} of $H^\infty$. We endow this space with the subspace topology of the weak\nobreakdash-* topology on the topological dual $(H^\infty)'$, which we will refer to as the \emph{Gelfand topology}. For each $z \in \D$ we consider the point-evaluation functional
\begin{equation*}
  \pi_z : H^\infty \to \C,\; f \mapsto f(z).
\end{equation*}
This is clearly multiplicative, bounded and linear and therefore belongs to $\Delta(H^\infty)$. The set of all such functionals $\pi_z, z \in \D$ will be denoted as $\Delta_0$. The \emph{corona} is defined as the complement of closure of $\Delta_0$ in the Gelfand topology. The corona theorem now states:

\begin{theorem}[L. Carleson] \label{thm:corona}
  The corona is empty. In other words, $\Delta_0$ is dense in $\Delta(H^\infty)$.
\end{theorem}

There is an equivalent version of the theorem, as given by the following proposition:

\begin{proposition} \label{prop:corona-equiv}
  $\Delta_0$ is dense in $\Delta(H^\infty)$ if and only if for any $\delta > 0$ and $f_1, \dots, f_n \in H^\infty$ such that $\sum_{j=1}^n \vert f_j(z) \vert \geq \delta, z \in \D$, there exist $g_1, \dots g_n \in H^\infty$ such that $\sum_{j=1}^n f_j g_j = 1$.
\end{proposition}

\begin{proof}
  Assume $\Delta_0$ is dense in $\Delta(H^\infty)$ and let $f_1, \dots, f_n \in H^\infty$, and $\delta > 0$ such that $\sum_{j=1}^n \vert f_j(z) \vert \geq \delta, z \in \D$. Denote by $I$ the ideal in $H^\infty$ generated by $f_1, \dots, f_n$. If $1 \in I$, then the assertion is established. Assume towards a contradiction that $I$ is a proper ideal, then there exists a maximal ideal $J \supset I$. Since $\Delta(H^\infty)$ is a commutative Banach algebra, there exists a $\phi \in \Delta(H^\infty)$ such that $J = \ker \phi$. Therefore we have $\phi(f_j) = 0$ for $j=1,\dots,n$. Since $\Delta_0$ is dense, there is a net $(\pi_{z_m})_{m \in M}$ in $\Delta_0$ such that $\pi_{z_m} \to \phi$ in the weak\nobreakdash-* topology, that is the net converges pointwise. Therefore, for all $j=1,\dots,n$ we have $f_j(z_m) = \pi_{z_m}(f_j) \to \phi(f_j) = 0$ and in particular
  $$ \lim_{m \in M} \sum_{j=1}^n \vert f_j(z_m) \vert = 0, $$
  a contradiction.

  For the other implication, assume towards a contradiction that $\Delta_0$ is not dense in $\Delta(H^\infty)$. Then there exists some $\phi_0 \in \Delta(H^\infty)$ and an open neighbourhood $U$ of $\phi_0$ such that $\Delta_0 \cap U = \emptyset$. Since the sets of the form
  $$ \{ \phi \in \Delta(H^\infty) : \vert (\phi - \phi_0)(f_j) \vert < \varepsilon, j = 1, \dots, n \}, $$
  for some $n \in \N, f_1, \dots, f_n \in H^\infty$ and $\varepsilon > 0$, form a neighbourhood basis of $\phi_0$ in the weak\nobreakdash-* topology, there exists a neighbourhood $V \subseteq U$ described by some $n \in \N, f_1, \dots, f_n \in H^\infty$ and $\delta > 0$. Define $\widetilde{f_j} \coloneqq f_j - \phi_0(f_j)$, for $j=1,\dots,n$, then clearly $\phi_0(\widetilde{f_j}) = 0$. Since $\Delta_0 \cap V = \emptyset$, for any $z \in \D$ we have $\pi_z \notin V$ and therefore there exists some $j_0 \in \{ 1, \dots, n \}$ such that,
  $$ \delta \leq \vert (\pi_z - \phi_0)(f_{j_0}) \vert = \vert f_{j_0}(z) - \phi_0(f_{j_0}) \vert = \vert \widetilde{f_{j_0}}(z) \vert. $$
  Since $\widetilde{f_j} \in H^\infty$ for $j=1,\dots,n$, and $ \sum_{j=1}^n \vert \widetilde{f_j}(z) \vert \geq \delta, $ there exist $g_1, \dots, g_n \in H^\infty$ such that $\sum_{j=1}^n \widetilde{f_j} g_j = 1$. But this yields
  $$ 1 = \phi_0(1) = \phi_0 \left( \sum_{j=1}^n \widetilde{f_j} g_j \right) = \sum_{j=1}^n \phi_0(\widetilde{f_j}) \phi_0(g_j) = 0, $$
  a contradiction.
\end{proof}

\section{First Steps}

Over the following sections we will prove a stronger version of the right statement in \Cref{prop:corona-equiv}:

\begin{theorem}
  There exist constants $C_{n,\delta}$ only depending on $n \in \N$ and $\delta > 0$, such that if $f_1, \dots f_n \in \Hol(\D)$ with
  $$ \Vert f_j \Vert_\infty \leq 1, \hspace{0.5em} j = 1, \dots, n, \quad \textrm{and} \quad \sum_{j=1}^n \vert f_j(z) \vert^2 \geq \delta, \hspace{0.5em} z \in \D, $$
  then there exist $g_1, \dots, g_n \in \Hol(\D)$ with
  $$ \Vert g_j \Vert_\infty \leq C_{n,\delta}, \hspace{0.5em} j = 1, \dots, n, \quad \textrm{and} \quad \sum_{j=1}^n f_j g_j = 1. $$
\end{theorem}

\begin{innerproof}
  We will give the proof in multiple steps.

  \textit{Step 1 (Reduction to $f_1, \dots, f_n \in \Hol(\cl{\D})$):} Assume that the statement of the theorem holds for all $\widetilde{f_1}, \dots, \widetilde{f_n} \in \Hol(\cl{\D})$, we claim that it then also holds in its original form\footnote{Note that this does \textbf{not} mean that we can assume $f_1, \dots, f_n \in \Hol(\cl{\D})$ in the previous proposition.}. For our given $f_1, \dots, f_n$ satisfying the premise of the theorem and all $0 < s < 1$ we define $ f_{j,s}(z) \coloneqq f_{j}(sz), \, j=1,\dots,n $. Then for every $0 < s < 1$ and $j = 1, \dots, n$ the function $f_{j,s}$ is in $\Hol(\cl{\D})$ and satisfies the premise of the theorem. By our assumption there exist $g_{j,s} \in H^\infty, \, j=1,\dots,n$ such that
  $$ \Vert g_{j,s} \Vert_\infty \leq C_{n,\delta}, \hspace{0.5em} j = 1, \dots, n, \quad \textrm{and} \quad \sum_{j=1}^n f_{j,s} g_{j,s} = 1. $$
  For a fixed $j \in \{ 1, \dots, n \}$, the set $\{ g_{j,s} : 0 < s < 1 \}$ is uniformly bounded and therefore normal in $\Hol(\D)$. By Montel's Theorem there exists a sequence $s_m \to 1$ and some $g_j \in \Hol(\D)$ such that $g_{j,s_m} \to g_{j}$ compactly. In particular, we obtain
  $$ \Vert g_j \Vert_\infty = \lim_{m \to \infty} \Vert g_{j,s_m} \Vert_\infty \leq C_{n,\delta}, \quad j=1,\dots,n, $$
  and
  $$ 1 = \lim_{m \to \infty} \sum_{j=1}^n f_{j,s_m} g_{j,s_m} = \sum_{j=1}^n f_j g_j, $$
  concluding our claim. We may thus assume that our given $f_1, \dots, f_n$ are holomorphic on $\cl{\D}$ instead.

  \textit{Step 2 (Solve in $C^\infty(\cl{\D})$):} For $j=1,\dots,n$ we define
  $$ h_j \coloneqq \frac{\bar{f_j}}{\sum_{k=1}^n \vert f_k \vert^2} \in C^\infty(\cl{\D}), $$
  then clearly $ \sum_{j=1}^n f_j h_j = 1$ and $\Vert h_j \Vert_\infty \leq \frac{1}{\delta}$. The real task now lies in changing the $h_j$ to become holomorphic in $\D$, without losing control over the boundedness of the solutions.
\end{innerproof}

\section{Wirtinger Derivatives}

Before we continue we want to briefly introduce a useful generalization of the complex derivative.

\begin{definition}
  Let $\Omega \subseteq \R^2$ be open. Then the \emph{Wirtinger derivatives} (or \emph{Wirtinger operators}) are defined on $C^1(\Omega)$ by
  \begin{equation*}
    \frac{\partial}{\partial z} \coloneqq \frac{1}{2} \left( \frac{\partial}{\partial x} - i \frac{\partial}{\partial y} \right), \quad \textrm{and} \quad
    \frac{\partial}{\partial \bar{z}} \coloneqq \frac{1}{2} \left( \frac{\partial}{\partial x} + i \frac{\partial}{\partial y} \right).
  \end{equation*}
  We will also abbreviate these operators as $\partial$ and $\bar{\partial}$, respectively.
\end{definition}

Note that by writing a complex number $z \in \C$ as $z = x + iy$ with $x, y \in \R$ we can identify $\C \cong \R^2$. Therefore we can also interpret the Wirtinger operators to act on $C^1(\Omega)$ with an open subset $\Omega \subseteq \C$.

Before listing properties of the Wirtinger operators we quickly want to recall that a function $f \in C^1(\Omega), f = u + iv$ is holomorphic if and only if it satisfies the \emph{Cauchy--Riemann equations}:
\begin{equation*}
  \frac{\partial u}{\partial x} = \frac{\partial v}{\partial y}, \quad \textrm{and} \quad \frac{\partial u}{\partial y} = -\frac{\partial v}{\partial x}.
\end{equation*}

\begin{remark}
  Let $\Omega \subseteq \C$ be open and $f \in C^1(\Omega)$.
  \begin{enumerate}
    \item The Wirtinger operators are $\C$-linear, satisfy the Leibniz rule\footnote{This means that the Wirtinger opeartors are derivatives from an algebraic perspective.} and
    $$ \overline{\left( \frac{\partial f}{\partial z} \right)} = \frac{\partial \bar{f}}{\partial \bar{z}}, \quad \overline{\left( \frac{\partial f}{\partial \bar{z}} \right)} = \frac{\partial \bar{f}}{\partial z} $$
    \item If $f \in \Hol(\Omega), f = u + iv$, then
    $$
      \frac{\partial f}{\partial z} = \frac{1}{2} \left( \frac{\partial f}{\partial x} - i \frac{\partial f}{\partial y} \right) = \frac{1}{2} \left( \frac{\partial u}{\partial x} + i\frac{\partial v}{\partial x} - i\frac{\partial u}{\partial y} + \frac{\partial v}{\partial y} \right) = \frac{\partial u}{\partial x} + i \frac{\partial v}{\partial x} = \frac{\partial f}{\partial x} = f'.
    $$
    \item Since
    \begin{align*}      
      \frac{\partial f}{\partial \bar{z}} &= \frac{1}{2} \left( \frac{\partial f}{\partial x} + i \frac{\partial f}{\partial y} \right) = \frac{1}{2} \left( \frac{\partial u}{\partial x} + i\frac{\partial v}{\partial x} + i\frac{\partial u}{\partial y} - \frac{\partial v}{\partial y} \right) = \\
      &= \frac{1}{2} \left( \frac{\partial u}{\partial x} - \frac{\partial v}{\partial y} \right) + \frac{i}{2} \left( \frac{\partial v}{\partial x} + \frac{\partial u}{\partial y} \right),
    \end{align*}
    we have that\footnote{This can be interpreted as ``$f$ is independant of $\overline{z}$''.} $f \in \Hol(\Omega)$ if and only if $\wirtzbar f = 0$.
    \item On $C^2(\Omega)$, the \emph{Laplace operator} can be represented as
    $$ \Delta = \frac{\partial^2}{\partial x^2} + \frac{\partial^2}{\partial y^2} = \left( \frac{\partial}{\partial x} - i \frac{\partial}{\partial y} \right) \left( \frac{\partial}{\partial x} + i \frac{\partial}{\partial y} \right) = 4 \frac{\partial}{\partial z} \frac{\partial}{\partial \bar{z}}. $$
  \end{enumerate}
\end{remark}

\begin{innerproof}[Proof (continued)]
  \textit{Step 3 (The Koszul complex):} We consider the spaces
  \begin{equation*}
    C_0 \coloneqq C^\infty(\overline{\D}), \quad C_1 \coloneqq (C_0)^n, \quad C_2 \coloneqq \{ A \in (C_0)^{n \times n} : A = -A^T \}
  \end{equation*}
  and the linear maps
  \begin{align*}
    P_{1,0} : C_1 \to C_0,\, (g_j)_{j=1}^n \,\mapsto\, \sum_{j=1}^n g_j f_j, \quad P_{2,1} : C_2 \to C_1,\, (g_{jk})_{j,k=1}^n \,\mapsto\, \left( \sum_{k=1}^n g_{jk} f_k \right)_{j=1}^n.
  \end{align*}
  Applying $\wirtzbar$ pointwise in $C_j, j = 0, 1, 2$, the resulting connections are visualized in the diagram below, called the \emph{Koszul complex}:
  \begin{center}
    \begin{tikzcd}[row sep=large,column sep=large]
      C_2 \arrow[r, "P_{2,1}"] \arrow[d, "\wirtzbar", swap] & C_1 \arrow[r, "P_{1,0}"] \arrow[d, "\wirtzbar", swap] & C_0 \arrow[d, "\wirtzbar", swap] \\
      C_2 \arrow[r, "P_{2,1}", swap] & C_1 \arrow[r, "P_{1,0}", swap] & C_0
    \end{tikzcd}
  \end{center}
\end{innerproof}

\begin{lemma} \label{lem:koszul}
  The Koszul complex has the following properties:
  \begin{enumerate}
    \item The diagram is commutative, that is we have $P_{j+1,j} \wirtzbar = \wirtzbar P_{j+1,j}$ for $j = 0,1$.
    \item The horizontal sequences are exact, that is $\ran P_{2,1} = \ker P_{1,0}$.
    \item The maps $\wirtzbar : C_j \to C_j$ for $j=0,1,2$ are surjective.
  \end{enumerate}
\end{lemma}

\begin{proof}{\ }
  \begin{enumerate}
    \item For $g \in C_0$ and $f \in \Hol(\cl{\D})$ we have
    $$ \frac{\partial (gf)}{\partial \bar{z}} = \frac{\partial g}{\partial \bar{z}} \cdot f + g \cdot \frac{\partial f}{\partial \bar{z}} = \frac{\partial g}{\partial \bar{z}} \cdot f $$
    and together with the linearity of $\wirtzbar$ the statement follows.
    \item ``$\subseteq$'': Let $g \in C_2, g = (g_{jk})_{j,k=1}^n$, then
    $$ P_{1,0} P_{2,1} g = P_{1,0} \left[ \left( \sum_{k=1}^n g_{jk} f_k \right)_{j=1}^n \right] = \sum_{j=1}^n \sum_{k=1}^n g_{jk} f_k f_j = 0 $$
    since $g$ is skew-symmetric and therefore $g \in \ker P_{1,0}$.

    ``$\supseteq$'' Let $g \in \ker P_{1,0} \subseteq C_1, g = (g_1, \dots, g_n)$. We define $p = (p_{jk})_{j,k=1}^n \in C_2$ by
    $$ p_{jk} \coloneqq \frac{1}{\displaystyle\sum_{\ell=1}^n \vert f_\ell \vert^2} (g_j \overline{f_k} - g_k \overline{f_j}). $$
    Then for any $j=1,\dots,n$ we have
    \begin{align*}
      (P_{2,1} p)_j &= \sum_{k=1}^n p_{jk} f_k = \frac{1}{\displaystyle\sum_{\ell=1}^n \vert f_\ell \vert^2} \sum_{k=1}^n (g_j \vert f_k \vert^2 - g_k \overline{f_j} f_k) = \\
      &= g_j - \frac{1}{\displaystyle\sum_{\ell=1}^n \vert f_\ell \vert^2} \overline{f_j} \sum_{k=1}^n g_k f_k = g_j - \frac{1}{\displaystyle\sum_{\ell=1}^n \vert f_\ell \vert^2} \overline{f_j} P_{1,0} g = \\
      &= g_j,
    \end{align*}
    and therefore $g_j \in \ran P_{2,1}$.
    \item For given $v \in C^\infty(\cl{\D})$ we want to solve the partial differential equation
    $$ \frac{\partial u}{\partial \bar{z}} = v \quad \textrm{(on $\cl\D$)} $$
    for some $u \in C^\infty(\cl{\D})$. We will approach this using a fundamental solution of the differential operator $\wirtzbar$. Recall that
    $$ \Gamma(z) \coloneqq \frac{1}{2 \pi} \log \vert z \vert $$
    is a fundamental solution of the Laplace operator, that is we have $\Delta \Gamma = \delta_0$ distributionally, where $\delta_0$ denotes the delta distribution at 0. We claim that $\frac{1}{\pi z}$ is a fundamental solution of $\wirtzbar$, and verify this via
    \begin{align*}
      \wirtzbar \frac{1}{z} &= \wirtzbar \frac{\bar{z}}{\vert z \vert^2} = \frac{1}{2} (\partial_x + i \partial_y) \frac{x - i y}{x^2 + y^2} = \\
      &= \frac{1}{2} \left[ \partial_x \frac{x}{x^2 + y^2} - i \partial_x \frac{y}{x^2 + y^2} + i \partial_y \frac{x}{x^2 + y^2} + \partial_y \frac{y}{x^2 + y^2} \right] = \\
      &= \frac{1}{2} \left[ \partial^2_x \log \vert z \vert + \partial^2_y \log \vert z \vert + i \left( \frac{2xy}{x^2 + y^2} - \frac{2xy}{x^2 + y^2} \right) \right] = \\
      &= \frac{1}{2} \Delta \log \vert z \vert = \frac{1}{2} 2 \pi \delta_0 = \pi \delta_0.
    \end{align*}
    Now let $\Omega \supset \cl{\D}$ be open such that $v \in C^\infty(\Omega)$ and choose $\varphi \in C_c^\infty(\Omega)$ such that $\restr{\varphi}{\cl{\D}} = 1$. Then $\varphi v \in C_c^\infty(\Omega)$, therefore
    $$ u(w) \coloneqq \left(\frac{1}{\pi z} \ast \varphi v\right)(w) = \frac{1}{\pi} \int_\Omega \frac{\varphi(z) v(z)}{w - z} \diff \lambda^2(z) $$
    is a classical solution of $\wirtzbar u = \varphi v$ in $\Omega$. Since $\varphi v = v$ on $\cl{\D}$, we get $\wirtzbar u = v$ on $\cl\D$, as desired.

    Arguing pointwise shows the surjectivity of the maps $\wirtzbar : C_\ell \to C_\ell$ for $\ell = 0, 1$. For $\ell = 2$ and given $b = (b_{jk})_{j,k=1}^n \in C_2$ we first solve
    $$ \wirtzbar a_{jk} = b_{jk}, \quad \textrm{for} \quad 1 \leq j < k \leq n $$
    and then set $a_{jj} = 0$ and $a_{jk} = -a_{kj}$ for $n \geq j > k \geq 1$.
  \end{enumerate}
\end{proof}

\begin{innerproof}[Proof (continued)]
  \textit{Step 4 (Apply to $h = (h_1, \dots, h_n) \in C_1$):} In step 2 we constructed an element $h = (h_1, \dots, h_n) \in C_1$ by setting
  $$ h_j \coloneqq \frac{\bar{f_j}}{\sum_{k=1}^n \vert f_k \vert^2}. $$
  By our construction we have $P_{1,0} h = 1$ and therefore $0 = \wirtzbar P_{1,0} h = P_{1,0} \wirtzbar h$, thus $\wirtzbar h \in \ker P_{1,0}$. By \Cref{lem:koszul} there exists $b \in C_2$ such that $P_{2,1} b = \wirtzbar h$ and $a \in C_2$ such that $\wirtzbar a = b$. We now set $g \coloneqq h - P_{2,1} a \in C_1$. Then
  $$ P_{1,0} g = P_{1,0} h - P_{1,0} P_{2,1} a = 1 $$
  and
  $$ \wirtzbar g = \wirtzbar h - \wirtzbar P_{2,1} a = \wirtzbar h - P_{2,1} b = 0. $$
  Therefore $g$ is a solution to
  $$ \sum_{k=1}^n f_k g_k = 1 $$
  in $\Hol(\cl{\D})$. However, we do not have an estimate on $\vert g_j \vert$ yet.
\end{innerproof}

\section{Hardy Spaces}

Let $\mu$ denote the Lebesgue measure on $\T = \{ z \in \C : \vert z \vert = 1 \}$, i.e. the measure such that for a function $f : \T \to \C$ it holds that
$$ \int_\T f \diff \mu = \int_{-\pi}^\pi f(e^{i \vartheta}) \diff \vartheta. $$
We define the $L^p(\T)$-norms via the \emph{normed} Lebesgue measure $\frac{1}{2 \pi} \mu$:
\begin{align*}
  &\Vert f \Vert_p \coloneqq \left( \frac{1}{2\pi} \int_\T \vert f \vert^p \diff \mu \right)^{1/p}, \quad \textrm{for } 1 \leq p < \infty, \quad \textrm{and} \quad \Vert f \Vert_\infty \coloneqq \mathrm{ess.\,sup\,} \vert f \vert.
\end{align*}
For $f \in L^1(\T)$ and $n \in \N$ we define the $n$-th \emph{Fourier coefficient} by
$$ \hat{f}(n) \coloneqq \frac{1}{2\pi} \int_{-\pi}^{\pi} f(e^{i \vartheta}) e^{-i n \vartheta} \diff \vartheta. $$

For $1 \leq p \leq \infty$ we define the \emph{Hardy space} $H^p$ as the set of all $f \in \Hol(\D)$ with $\Vert f \Vert_p < \infty$, where
$$ \Vert f \Vert_p \coloneqq \lim_{r \to 1} \left( \frac{1}{2 \pi} \int_{-\pi}^{\pi} \vert f(r e^{i \vartheta}) \vert^p \diff \vartheta \right)^{1/p} \quad \textrm{for } p < \infty, \quad \textrm{and} \quad \Vert f \Vert_\infty \coloneqq \sup_{z \in \D} \vert f(z) \vert. $$
It is of note that convergence in the Hardy spaces implies compact convergence.

We define $H_0^p$ as the (closed) subspace of all $f \in H^p$, for which $f(0) = 0$.

We summarize the characterisation of Hardy spaces:
\begin{theorem} Let $1 \leq p \leq \infty$. Then:
  \begin{enumerate}
    \item $H^p$ is a Banach space\footnote{In particular, $H^\infty$ is a Banach algebra, which we already used in the introduction.}.
    \item For $p \leq q \leq \infty$ it holds that $H^p \supseteq H^q$.
    \item Let $f \in H^p$, then for almost all $e^{i \vartheta} \in \T$ the limit
    $$ \lim_{r \to 1} f(re^{i \vartheta}) \eqqcolon f^*(e^{i \vartheta}) $$
    exists and defines a function in $L^p(\T)$, also called the \emph{boundary values} of $f$.
    \item The map ${}^* : f \mapsto f^*$ is an isometry from $H^p$ onto
    $$ L^p_+(\T) \coloneqq \{ f \in L^p(\T) : \forall n < 0: \hat{f}(n) = 0 \}, $$
    which is a closed subspace of $L^p(\T)$.
    \item Every $f \in H^p$ can be written as a Cauchy integral of its boundary values:
    $$ f(z) = \frac{1}{2 \pi i} \int_\T \frac{f^*(\zeta)}{\zeta - z} \diff \mu(\zeta), \quad z \in \D $$
  \end{enumerate}
\end{theorem}

\begin{lemma}
  The map
  \begin{equation*}
    \Phi : L^\infty(\T) / (H^\infty)^* \to ((H_0^1)^*)',\, f + (H^\infty)^* \mapsto \left[ g \mapsto \frac{1}{2 \pi} \int_\T f g \diff \mu \right]
  \end{equation*}
  is an isometric isomorphism.
\end{lemma}

\begin{proof}
  We have $L^\infty(\T) \cong L^1(\T)'$ via the duality
  $$ \langle f, g \rangle \coloneqq \frac{1}{2\pi} \int_\T f g \diff \mu, \quad f \in L^\infty(\T),\; g \in L^1(\T). $$
  Since $(H_0^1)^* \leq L^1(\T)$ we therefore have $((H_0^1)^*)' \cong L^\infty(\T) / ((H_0^1)^*)^\perp$ via
  $$ \langle f, g \rangle = \frac{1}{2\pi} \int_\T f g \diff \mu, \quad f \in L^\infty(\T) / ((H_0^1)^*)^\perp,\; g \in (H_0^1)^*. $$
  It remains to show that $((H_0^1)^*)^\perp = (H^\infty)^*$. Let $w^* \in ((H_0^1)^*)^\perp \leq L^\infty(\T)$, then for any $n \in \N$ we have
  $$ 0 = \langle w^*, (z^n)^* \rangle = \langle w^*, e^{i n t} \rangle = \widehat{w^*}(-n). $$
  Therefore $w^* \in L_+^\infty(\T) = (H^\infty)^*$. For the other inclusion let $w^* \in (H^\infty)^*$ and $h^* \in (H_0^1)^*$, then
  \begin{align*}
    \langle w^*, h^* \rangle &= \frac{1}{2\pi} \int_0^{2 \pi} w^*(e^{i \vartheta}) h^*(e^{i \vartheta}) \diff \vartheta = \lim_{r \to 1} \frac{1}{2\pi} \int_0^{2 \pi} w(r e^{i \vartheta}) h(r e^{i \vartheta}) \diff \vartheta = \\
    &= \lim_{r \to 1} \frac{1}{2\pi i} \int_\T \frac{w_r(\zeta) h_r(\zeta)}{\zeta} \diff \zeta = \lim_{r \to 1} w_r(0) h_r(0) = 0,
  \end{align*}
  where the second-to-last equation follows since $w_r h_r \in \Hol(\cl{\D})$ and the last one since $h(0) = 0$. Therefore $w^* \in ((H_0^1)^*)^\perp$, concluding the proof.
\end{proof}

\begin{innerproof}[Proof (continued)]
    \textit{Step 5 (Dualisation):} Applying the above lemma to our previous situation we can re-describe the norm of $v + (H^\infty)^*$:
    \begin{align*}
      \Vert v + (H^\infty)^* \Vert_\infty &= \Vert \Phi(v + (H^\infty)^*) \Vert = \\
      &= \sup \{ \vert \Phi(v + (H^\infty)^*)(g) \vert : f \in (H_0^1)^*, \Vert f \Vert_1 \leq 1 \} = \\
      &= \sup \left\{ \left\vert \frac{1}{2 \pi} \int_\T v f^* \diff \mu \right\vert : f \in H_0^1, \Vert f \Vert_1 \leq 1 \right\} = \\
      &= \sup \left\{ \left\vert \frac{1}{2 \pi} \int_\T v f \diff \mu \right\vert : f \in \Hol(\cl{\D}), f(0) = 0, \Vert f \Vert_1 \leq 1 \right\}
    \end{align*}
\end{innerproof}

\begin{proof}
\end{proof}

\begin{innerproof}[Proof (continued)] \textit{Step 6:} todo

  We want to redescribe the integral
  $$ \frac{1}{2\pi} \int_\T v f \diff \mu, $$
  where $v \in L^\infty(\T)$ and $f \in \Hol(\cl{\D}), f(0) = 0, \Vert f \Vert_1 \leq 1$. Let $\sigma \coloneqq v f$ and
  $$ \varphi(z) \coloneqq \frac{1}{2\pi} \log \vert z \vert, \quad z \neq 0. $$
  By Green's second identity we have
  \begin{align*}
    \int_\D \sigma \Delta \varphi - \varphi \Delta \sigma \diff \lambda^2 = \int_\T \sigma \frac{\partial \varphi}{\partial r} - \varphi \frac{\partial \sigma}{\partial r} \diff \mu,
  \end{align*}
  where $\frac{\partial}{\partial r}$ denotes the radial derivative.
  Since $\restr{\varphi}{\T} = 0$ and $\restr{\frac{\partial \varphi}{\partial r}}{\T} = \frac{1}{2 \pi}$ the right integral equals
  $$ \frac{1}{2 \pi} \int_\T v f \diff \mu $$
  Since distributionally $\Delta \varphi$ equals $\delta_0$, the delta distribution at 0, we have
  $$ \int_\D \sigma \Delta \varphi \diff \lambda^2 = \sigma(0) = v(0) f(0) = 0. $$
  Finally,
  $$ \Delta \sigma = \Delta (v f) = 4 \wirtz \wirtzbar (v f) = 4 \wirtz (v \wirtzbar f + f \wirtzbar v) = 4\wirtz(f u) = 4(f \partial u + u f'). $$
  With $\psi \coloneqq -\varphi = \frac{1}{2\pi} \log \frac{1}{\vert z \vert}$ we thus obtain
  \begin{align*}
    \frac{1}{2 \pi} \int_\T v f \diff \mu &= 4 \int_\D \psi \left( f \partial u + u f' \right) \diff \lambda^2 = \\ 
    &= 4 \left( \int_\D f \, \partial u \, \psi \diff \lambda^2 + \int_\D u f' \psi \diff \lambda^2 \right) \eqqcolon 4 (I_1 + I_2).
  \end{align*}
  Our goal is to show the existence of a constant $C_{n,\delta}$ such that
  $$ \left\vert \frac{1}{2 \pi} \int_\T v f \diff \mu \right\vert \leq 4 (\vert I_1 \vert + \vert I_2 \vert) \leq C_{n,\delta}. $$
\end{innerproof}

\section{Integral estimates}

\begin{lemma}
  Let $f, g, u, v \in \Hol(\cl{\D})$, then the following integral estimates hold:
  \begin{enumerate}
    \item $ \displaystyle \int_\D \vert f' \vert^2 \psi \diff \lambda^2 \leq ? \Vert f \Vert_2^2 $
    \item $ \displaystyle \int_\D \vert f g' \vert \log \frac{1}{\vert z \vert} \diff \lambda^2 \leq 2 \pi \Vert f \Vert_2^2 \Vert g \Vert_\infty $
    \item $ \displaystyle \int_\D \vert f g u' v' \vert \log \frac{1}{\vert z \vert} \diff \lambda^2 \leq 2 \pi \Vert f \Vert_2 \Vert g \Vert_2 \Vert u \Vert_\infty \Vert v \Vert_\infty $
    \item $ \displaystyle \int_\D \vert f u' v' \vert \log \frac{1}{\vert z \vert} \diff \lambda^2 \leq 2 \pi \Vert f \Vert_1 \Vert u \Vert_\infty \Vert v \Vert_\infty $
    \item $ \displaystyle \int_\D \vert f g' u' \vert \log \frac{1}{\vert z \vert} \diff \lambda^2 \leq \pi \Vert f \Vert_2 \Vert g \Vert_2 \Vert u \Vert_\infty $
    \item $ \displaystyle \int_\D \vert f' u' \vert \log \frac{1}{\vert z \vert} \diff \lambda^2 \leq 2 \pi \Vert f \Vert_1 \Vert u \Vert_\infty $
  \end{enumerate}
\end{lemma}

\begin{proof} {\ }
  \begin{enumerate}
    \item Applying Green's formula on $f \bar{f}$ and $\psi$ yields
    $$ \int_\D \psi \Delta (f \bar{f}) - f \bar{f} \Delta \psi \diff \lambda^2 = \int_\T \psi \frac{\partial}{\partial r} (f \bar{f}) - f \bar{f} \frac{\partial}{\partial r} \psi \diff \mu, $$
    or, simplified,
    $$ \int_\D \psi \Delta (f \bar{f}) \diff \lambda^2 + \vert f(0) \vert^2 = \int_\T \vert f \vert^2 \diff \mu = \Vert f \Vert_2^2 $$

    Since
    \begin{align*}
      \Delta(f \bar{f}) &= 4 \partial \wirtzbar (f \bar{f}) = 4 \partial ( \bar{f} \cdot \wirtzbar f + f \cdot \wirtzbar \bar{f} ) = 4 \partial (f \cdot \wirtzbar \bar{f}) = 4 (\partial f \cdot \wirtzbar \bar{f} + f \cdot \wirtzbar \partial \bar{f}) = \\
      &= 4 (\partial f \cdot \overline{\partial f} + f \cdot \wirtzbar \,\overline{ \wirtzbar f }) = 4 \partial ( f \wirtzbar \bar{f} ) = 4 \vert f' \vert^2
    \end{align*}
    and $ \vert f(0) \vert^2 \geq 0$ we obtain
    $$ \frac{2}{\pi} \int_\D \vert f' \vert^2  \log \frac{1}{\vert z \vert} \diff \lambda^2 \leq \Vert f \Vert_2^2 $$
    and rearranging yields the desired inequality.
  \end{enumerate}
\end{proof}

\end{document}